
\documentclass[11pt]{article}
%\linespread{1.1}
\usepackage{graphicx,import}
\usepackage{amsmath}
\usepackage{setspace}
\usepackage{color}
\usepackage{float}
\usepackage[inkscapelatex=false,inkscapearea=page]{svg}
\onehalfspacing
%\doublespacing

\title {Architecture and Open Queue Model \\ \bigskip \large Thesis}
\author {Silvio Dei Giudici, Marco Morella, Mattia Nicolella}
\date{26 September 2020}

\begin{document}
\maketitle
\section{Introduction}
This section will deal with the assumptions we have made so far and why we've made them. Alternatives will be described for the various possibilities.\\
\subsection{Assumptions}
\begin{enumerate}
\item \textbf{Assumption 2}: The only entity needing a storage system is the Central Node, in particular we are assuming that all other nodes receive information that can be processed, stored and aggregated in the RAM which will never be full. 
\item \textbf{Assumption 3}: more of a design choice for the moment, each node is modeled as a simple queue composed of CPU+RAM together, in future they will be extended and may need more component and a review of the entire queue model.
\item \textbf{Assumption 4}: In this first stage we won't be holding account of the fault tolerance of the system, in particular we are not considering duplication of the data not in storaging neither in multiple link sending.
\end{enumerate}

\section{System functioning and message flow}
The system is composed by:
\begin{itemize}
\item Central Node: core of the operations, all transitions-command couples will arrive to it to be logged, all telemetries must arrive to him then be aggregated, analyzed and stored on its disk. It manages all regional nodes. When a transition message reaches the central node it will need to generate a command in response.
\item Regional Node: transitions that need a regional level command will reach this and all transitions-command couples fixed in local level will be forwarded here to be sent in batch to the central node. Telemetries will flow through here to be aggregated and sent to the central node. A regional node manages a subset of all local nodes, disjoint from any other regional node domain.
\item Local Node: all transitions needing a local fix will go through them and be forwarded as a transition-command couple to the regional node. Also all transitions needing a regional or central level fix will be forwarded from here. A local node manages a subset of all actuators and sensors, disjoint from any other local node.
\item Actuator: commands will reach the actuators for a state change in the system, they can also send transitions to the higher levels when their state is changed without receiving a command message .
\item Sensor: periodically will generate telemetry messages and send them to the local level to be forwarded up to the central node, also will generate and send transition messages when their state is changed. They only send, never receive.
\end{itemize}
Both local and regional nodes need an acknowledgement reply from the higher level, respectively regional and central nodes. \\
All aggregations we talked about are determinated by parameters that will be given.\\
There is no duplication in sent messages.\\ 
All message types will be throughly analyzed in the Classes section.\\

\section{Structural scheme}
This section will explain how the structural scheme for the system is made and its project's choices.\\
In the model we are going to show, we used as connection between the upper level node and the lower level ones consists of a WAN connecting the upper level router to all lower level ones.\\
\begin{figure}[H]
	\hspace*{-3.75cm}
	\centering
  \frame{\includesvg[width=20cm]{structural_scheme_central_regional}}
  \caption{Focus on Central and Regional nodes}
\end{figure}
\begin{figure}[H]
	\hspace*{-3.75cm}
	\frame{\includesvg[width=20cm]{structural_scheme_local}}
	\caption{Focus on Regional and Local nodes}
\end{figure}
Figure 1 represents the connection between Central and Regional nodes, while Figure 2 shows the connection between Regional and Local. We can  see that lower level router is used as an interface for lower level nodes to receive and send messages to the upper level WAN which then forwards to the node. \\
\begin{figure}[H]
	\hspace*{-3.75cm}
  \frame{\includesvg[width=20cm]{structural_scheme_local_sensor}}
  \caption{Focus on Local nodes and end points}
\end{figure}
Instead in figure 3, which displays the connection between Local nodes and the end points of the edge system, namely actuators and sensors, one LAN is used for each type of connection technology, in this case MANET and WIRELESS but more types can be added, both spanning between end points and their respective upper level node.\\
The whole scheme:
\begin{figure}[H]
	\hspace*{-3.75cm}
	\frame{\includesvg[width=20cm]{structural_scheme_complete}}
  \caption{Structural scheme}
\end{figure}

\section{Queues Scheme and Formulas}
The following open queue scheme will be divided in segments during the analysis of each queue.
\begin{figure}[H]
	\vspace*{-0.5cm}
	\hspace*{-3.5cm}
	\centering
	\frame{\includesvg[width=25cm, height=19cm]{queue_network_wan}}
	\caption{The open queue scheme}
\end{figure}
\subsection{Classes}
Based on the specifics and the architecture, we opted for a multi-class open queueing model, in particular at the moment we have three different classes going through the system:
\begin{itemize}
\item t: telemetry events, where a sensors sends to its superior the data it has periodically, this data then will be aggregated at each level up until the central node.
\item e: transition events, which are sent by actuators or sensors when their state changes.
\item c: command events, generated by local, regional or central nodes, in response to a transition message which reached it from an actuator or sensor.
\item bs: batch send, generated by the regional nodes periodically to send to the disk aggregated data about the transition-command pairs.
\end{itemize}
Another type of message flowing through the network is a reply message but since they are triggered only by transition events we aggregated them in the visits.
Due to the nature of this kind of open queueing model, the formulas regarding Utilization factor, system demand and response time are very similar for the components.

\subsection{Notation}
We will deline the notation in the following formulas:
Since the structure of the network is very different between the branches of the tree, we need a notation that express how many (and which) nodes, sensors and actuators we have in each branch:
\begin{itemize}
\item $R$, is the set of regional nodes in the network. Every element in the set is express through an unique number.
\item $L^{\{i\}}$, is the set of local nodes that have in common the same regional node with identifier $i$.
\item $LAN^{\{i,j\}}$, is the set of lan's types that the local node $j$ (which belongs to the branch of regional node $i$) is connected to.
\item $S^{\{i,j,k\}}$, is the set of sensors' types that are placed in the LAN $k$ (which belongs to the $j$ local node in the $i$ regional node).
\item $A^{\{i,j,k\}}$, is the set of actuators' types that are placed in the LAN $k$ (which belongs to the $j$ local node in the $i$ regional node).
\item $S^{\{i,j,k\}}_{t}$, is the set of sensors which are of type $t$, placed in the LAN $k$ (which belongs to the $j$ local node in the $i$ regional node).
\item $A^{\{i,j,k\}}_{t}$, is the set of actuators which are of type $t$, placed in the LAN $k$ (which belongs to the $j$ local node in the $i$ regional node).
\end{itemize}
Now for the parameters we are going to use:
\begin{itemize}
\item $\lambda_t$, $\lambda_e$, $\lambda_c$: $\lambda$ is the arrival rate, respectively for the data type telemetry, transition and command.
\item $V$: mean number of visits.
\item $S$: service time.
\item $D$: service demand.
\item $U$: utilization factor.
\item $R$: response time.
\end{itemize}
The elements we are going to analyze are:
\begin{itemize}
\item cn : central node.
\item cns : central node storage
\item rn : regional node.
\item ln : local node.
\item wl : wireless.
\item mt : manet.
\end{itemize}
We will navigate the queue network starting from the sensors and actuators until reaching the central node.

\begin{figure}[H]
	\centering
	\frame{\includesvg[width=20cm, height=17cm]{queue_network_wireless_manet}}
	\caption{Focus on actuators, sensors,  local nodes, wireless and MANET LANs}
\end{figure}

\subsection{Actuators and Sensors}
Sensors at the moment don't receive any command, we still have drawn them as queues for future extensions. Conversely, actuators are modeled as queues since they can receive command from the central node or from the primary/secondary node where they are connected in response to a transition event.\\
Regarding the sensors, $\lambda_{s,t}$ and $\lambda_{s,e}$ are the rates at which the sensor sends a telemetry or a transition message respectively. The same is valid for the transition events rate of the actuators with $\lambda_{a,e}$.\\
We will now analyze the behaviour of an actuator which receives commands from its primary node though a wireless or MANET LAN. The same relations apply also if the actuator is connected to a secondary node.\\
The mean number of visits for a command message is 1.
The rate at which an actuator can receive a command ($\lambda_{a,c}$) is composed by the rate at which a command can be received from the central node and by the probability to receive a command message from the primary/secondary node in response to a transition message. We assumed that commands from the central node are sent with a rate of $\lambda_{cn,c}$ and can reach any actuator in the whole network with the same probability by Assumption 5, so we need to weight the rate of the central node issuing commands with all the possible paths that this command can follow (over the possible $n$ regional nodes, their primary and secondary nodes, $m+k$ and over all the actuators connected to one of these primary/secondary nodes, $a_{wl}+a_{mt}$). The probability of issuing a command in response to a transition message received by the upper primary/secondary node (with a rate of $\lambda_{pn,e}$) is $prb$ and must be weighted by all the possible actuators connected to that node, $a_{mt}+a_{wl}$, which can be contacted by that message.


%NEWWWW PARTTTTTT
Before going to the main formula we need to do some pre-computation.
%Let's suppose that we are in a branch of the system in which we belong to the $i$ regional node, $j$ local node and $k$ LAN.

%Firstly let's compute the number of actuators in this branch for each type $t$ of actuator:
%\begin{equation}
%    a_{t} = \sum_{\forall m,n,s \ | \ m \in R \ \land \ n \in L^{\{m\}} \ \land \ s \in LAN^{\{m, n\}}}{ | A^{\{m, n, s\}}_{t} | }
%\end{equation}

Firstly we need the set of actuators' types in all the system:
%Se volete in questa parte dite direttamente a parole che T è l'insieme di tutti i tipi di attuatori del sistema
\begin{equation}
    T = t \ | \ (\forall m,n,s \ | \ m \in R \ \land \ n \in L^{\{m\}} \ \land \ s \in LAN^{\{m, n\}} \land t \in A^{\{m,n,s\}})
\end{equation}

Then let's compute the number of actuators in all the system's branches for each type $t \in T$ of actuator in the system:
\begin{equation}
    a'_{t} = \sum_{\forall m,n,s \ | \ m \in R \ \land \ n \in L^{\{m\}} \ \land \ s \in LAN^{\{m, n\}}}{ |A^{\{m, n, s\}}_{t}| }
\end{equation}


Also we need to denote the probability of each type of actuator to receive a command. Indeed, the probability to receive a command it's not evenly distributed over the actuators' types, so there may be the case that a type of actuator receives more commands with respect to another type.
For this reason we denote $p_t$ the probability associated to the $t$ type of actuator.
Since when a command is generated it must be delivered to at least an actuator, we need to specify that the sum of all type's probabilities is 1:
\begin{equation}
    \sum_{t \in T}{p_t} = 1
\end{equation}

Now we are ready to do the real computation. Denoting $w \in T$ as the type of the actuator that we are considering, we can write:

\begin{equation}
	\begin{array}{l}
		V_{a,c} = 1 \\
        \lambda_{a, c} = prob_{cmd} * (\lambda_{cn, c} + \lambda_{rn, c} + \lambda_{ln, c}) * \frac{p_{w}}{\sum\limits_{t \in T}{a'_{t} * p_{t}}}\\
	\end{array}
\end{equation}

From the previous formulas and the open queue model we can compute the Service Demand, Utilization factor and Response time.
\begin{equation}
	\begin{array}{l}
		D_{a, c} = V_{a, c} * S_{a, c} \\
		U_{a, c} = \lambda_{a, c} * D_{a, c} \\
		R_{a, c} = \frac{D_{a, c}}{1 - U_{a,c}} \\
	\end{array}
\end{equation}

%END NEW PART


\subsection{WIRELESS/MANET Lan}
We are not considering the implication of the different technology, since the difference between them is only the service time of the network. Thus we will only show the formulas for Wireless and the Manet's formulas will be mirrored.\\
For the formulas we used the assumption 5 on equiprobability and the table presented in section 1.2, in particular the first four columns.\\
The mean number of visits is trivially one for telemetry and commands since only one messages goes through the LAN, it's two for transition since a reply is needed.\\
More interesting are the formulas regarding the arrival rates.\\
In the telemetry cases the total arrival rate $\lambda_{wl, t}$ is the rate of a single sensor multiplied by the number of sensors.\\
For transition instead, $\lambda_{wl, e}$ will be a sum of two terms, since both actuators and sensors send messages.\\
Finally for $\lambda_{wl, c}$, the first term is the case where the central node issues a command, so the sent command of that class of messages will arrive to an actuator with a certain probability, given the assumption 5 this probability is one over the sum of all possible paths that can be taken by this message: one of the regional node, then one of the primary or secondary node and then one of all the actuators, in this case we are concerned only if they go on the WIRELESS($\frac{a_{wl}}{a_{wl}+a_{mt}}$). The second term instead is the case where a command is prompted due to a transition message; so we multiply the arrival rate of transition event for primary or secondary node ($\lambda_{pn,e}$ or $\lambda_{sn,e}$ the LAN is connected to a secondary node) by the probability of the transition to generate a command ($prb$), multiplied by the probability that the message needs to reach an actuator in this network ($\frac{a_{wl}}{a_{wl}+a_{mt}}$). We are using Assumption 5 so all the actuators have the same probability to be contacted. It is a strong assumption and may need to be relaxed in future work.\\


%NEW PARTTTTTT
Before going to the main formula we need to do some pre-computation.
Let's suppose that we are in a branch of the system in which we belong to the $i$ regional node, $j$ local node and $k$ LAN.

Firstly we need the set of actuators' types in all the system:
%Se volete in questa parte dite direttamente a parole che T è l'insieme di tutti i tipi di attuatori del sistema
\begin{equation}
    T = t \ | \ (\forall m,n,s \ | \ m \in R \ \land \ n \in L^{\{m\}} \ \land \ s \in LAN^{\{m, n\}} \land t \in A^{\{m,n,s\}})
\end{equation}

Now let's compute the number of actuators in this branch for each type $t \in T$ of actuator:
\begin{equation}
    a_{t} = | A^{\{i, j, k\}}_{t} | 
\end{equation}

And this is the set of types on this branch:
\begin{equation}
    T' = | A^{\{i, j, k\}} | 
\end{equation}



Then let's compute the number of actuators in all the system's branches for each type $t \in T$ of actuator in the system:
\begin{equation}
    a'_{t} = \sum_{\forall m,n,s \ | \ m \in R \ \land \ n \in L^{\{m\}} \ \land \ s \in LAN^{\{m, n\}}}{ |A^{\{m, n, s\}}_{t}| }
\end{equation}


Also we need to denote the probability of each type of actuator to receive a command. Indeed, the probability to receive a command it's not evenly distributed over the actuators' types, so there may be the case that a type of actuator receives more commands with respect to another type.
For this reason we denote $p_t$ the probability associated to the $t$ type of actuator.
Since when a command is generated it must be delivered to at least an actuator, we need to specify that the sum of all type's probabilities is 1:
\begin{equation}
    \sum_{t \in T}{p_t} = 1
\end{equation}

Now we are ready to do the real computation. Denoting with $\lambda^{\{w\}}_{s, t}$ the arrival rate from the sensor of type $w \in T$: 


\begin{equation}
    \begin{array}{l}
        V_{wl, t} = 1 \\
        V_{wl, e} = 2 \\ %reply
        V_{wl,c} = 1 \\
        \lambda_{wl, t} = \sum\limits_{w \in T}{| S^{\{i,j,k\}}_{w} | * \lambda^{\{w\}}_{s, t}}  \\
        \lambda_{wl, e} = \sum\limits_{w \in T}{| S^{\{i,j,k\}}_{w} | * \lambda^{\{w\}}_{s, e}} + \sum\limits_{w \in T}{| A^{\{i,j,k\}}_{w} | * \lambda^{\{w\}}_{s, e}}\\
		\lambda_{wl, c} = prob_{cmd} * (\lambda_{cn, c} + \lambda_{rn, c} + \lambda_{ln, c}) * \frac{\sum\limits_{t \in T'}{a_{t} * p_{t}}}{\sum\limits_{t \in T}{a'_{t} * p_{t}}}  \\\


    \end{array}
\end{equation}



%END NEW PART




In the last equation, the second term requires a change from primary node arrival rate to secondary node arrival rate in the case that the upper node is indeed a secondary.\\
From the previous formulas and the open queue model we can compute the Service Demand, Utilization factor and Response time.
\begin{equation}
    \begin{array}{l}
        D_{wl, t} = V_{wl, t} * S_{wl, t} \\
        D_{wl, e} = V_{wl, e} * S_{wl, e} \\
        D_{wl, c} = V_{wl, c} * S_{wl, c} \\
        U_{wl, t} = \lambda_{wl, t} * D_{wl, t} \\
        U_{wl, e} = \lambda_{wl, e} * D_{wl, e} \\
        U_{wl, c} = \lambda_{wl, c} * D_{wl, c} \\
        U_{wl} = U_{wl, t} + U_{wl, e} + U_{wl, c} \\
        R_{wl, t} = \frac{D_{wl, t}}{1 - U_{wl}} \\
        R_{wl, e} = \frac{D_{wl, e}}{1 - U_{wl}} \\
        R_{wl, c} = \frac{D_{wl, c}}{1 - U_{wl}} \\
    \end{array}
\end{equation}

\subsection{Primary Node}
All formulas in this section are relatable to the previous one and thus we won't need any note on them.\\
Formulas for secondary nodes are the same as primary ones for now so we will only discuss these one.\\
The equations of the previous element will be used as arrival rates for the primary node thus both MANET and WIRELESS will be referred to.\\
While transitions and telemetries are simply the sum of the rates coming from MANET and WIRELESS, commands need an explanation.\\
Indeed the first term is the rate arriving to a single primary node thus divided by the number of regionals and then by the number of primary plus secondary nodes. The second term is again the probability of dependent commands.
\begin{equation}
    \begin{array}{l}
        V_{pn, t} = 1 \\
        V_{pn, e} = 2 \\ %reply
        V_{pn,c} = 1 \\
        \lambda_{pn, t} = \lambda_{wl, t} + \lambda_{mt, t} \\
        \lambda_{pn, e} = \lambda_{wl, e} + \lambda_{mt, e} \\
        \lambda_{pn, c} = \frac{1}{n} * (\frac{1}{m+k}) * \lambda_{cn, c} + prb * \lambda_{pn, e}  \\\
    \end{array}
\end{equation}
From the previous formulas and the open queue model we can compute the Service Demand, Utilization factor and Response time.
\begin{equation}
    \begin{array}{l}
        D_{pn, t} = V_{pn, t} * S_{pn, t} \\
        D_{pn, e} = V_{pn, e} * S_{pn, e} \\
        D_{pn, c} = V_{pn, c} * S_{pn, c} \\
        U_{pn, t} = \lambda_{pn, t} * D_{pn, t} \\
        U_{pn, e} = \lambda_{pn, e} * D_{pn, e} \\
        U_{pn, c} = \lambda_{pn, c} * D_{pn, c} \\
        U_{pn} = U_{pn, t} + U_{pn, e} + U_{pn, c} \\
        R_{pn, t} = \frac{D_{pn, t}}{1 - U_{pn}} \\
        R_{pn, e} = \frac{D_{pn, e}}{1 - U_{pn}} \\
        R_{pn, c} = \frac{D_{pn, c}}{1 - U_{pn}} \\
    \end{array}
\end{equation}

\subsection{Primary LAN}

\begin{figure}[H]
	\hspace*{-3.75cm}
	\centering
	\frame{\includesvg[width=20cm, height=17cm]{queue_network_primary_lan}}
	\caption{Focus on primary/secondary WAN, primary/secondary LAN and primary/secondary nodes}
\end{figure}

The primary LAN element of the system will deal with messages having as sender or receiver one of the $m$ primary nodes, thus all rates will be multiplied by $m$.\\
Furthermore in the telemetry case, since the primary nodes do a first step of data aggregation, the arrival rate on the primary LAN will be divided by the aggregation rate of the primary nodes $aggr_{pn}$. \\
Lastly, a command passes through the primary LAN if it is directed to one of the nodes that it manages, the probability that a message arrives then is the probability that it goes through its regional $\frac{1}{n}$ multiplied the probability that the message is for a primary instead of a secondary $\frac{m}{m+k}$.\\
Regarding the secondary LAN the same formulas will be applied for now, instead of pl we will have sl and the fraction for command will be $\frac{k}{m+k}$.
\begin{equation}
    \begin{array}{l}
        V_{pl, t} = 1 \\
        V_{pl, e} = 2 \\ %reply
        V_{pl,c} = 1 \\
        \lambda_{pl, t} = m*\frac{\lambda_{pn, t}}{aggr_{pn}} \\
        \lambda_{pl, e} = m*\lambda_{pn, e} \\
        \lambda_{pl, c} = \frac{1}{n} * (\frac{m}{m+k}) * \lambda_{cn, c}  \\\
    \end{array}
\end{equation}
From the previous formulas and the open queue model we can compute the Service Demand, Utilization factor and Response time.
\begin{equation}
    \begin{array}{l}
        D_{pl, t} = V_{pl, t} * S_{pl, t} \\
        D_{pl, e} = V_{pl, e} * S_{pl, e} \\
        D_{pl, c} = V_{pl, c} * S_{pl, c} \\
        U_{pl, t} = \lambda_{pl, t} * D_{pl, t} \\
        U_{pl, e} = \lambda_{pl, e} * D_{pl, e} \\
        U_{pl, c} = \lambda_{pl, c} * D_{pl, c} \\
        U_{pl} = U_{pl, t} + U_{pl, e} + U_{pl, c} \\
        R_{pl, t} = \frac{D_{pl, t}}{1 - U_{pl}} \\
        R_{pl, e} = \frac{D_{pl, e}}{1 - U_{pl}} \\
        R_{pl, c} = \frac{D_{pl, c}}{1 - U_{pl}} \\
    \end{array}
\end{equation}
Regarding the WAN PRIMARY, it forwards all the messages received from the LAN PRIMARY-SECONDARY directly to the LAN PRIMARY. The LAN PRIMARY needs to forward all the messages received from the primary nodes towards the WAN PRIMARY; so we will have the same rate and number of visits per message. The only parameter that can be different between these two LANs is the service time.
The same formulas can also be applied to the WAN SECONDARY and the LAN SECONDARY using the parameters that characterize these network.

\subsection{LAN Primary-Secondary}

\begin{figure}[H]
	\hspace*{-3.75cm}
	\centering
	\frame{\includesvg[width=20cm, height=17cm]{queue_network_primary_secondary_lan}}
	\caption{Focus on regional nodes, LAN primary-secondary, primary WAN and secondary WAN}
\end{figure}

The Primary-Secondary LAN will relay messages to and from primary and secondary LANs.\\
Telemetry and transition rate is the sum of the same rates for the lower level nodes, while for the command events, this LAN will receive $\frac{1}{n}$ command messages from the central, only the ones for its upper level regional node to be forwarded to the correct primary or secondary node.\\
\begin{equation}
    \begin{array}{l}
        V_{ps, t} = 1 \\
        V_{ps, e} = 2 \\ %reply
        V_{ps,c} = 1 \\
				\lambda_{ps, t} = \lambda_{pl, t} + \lambda_{sl,t} \\
				\lambda_{ps, e} = \lambda_{pl, e} + \lambda_{sl,e} \\
				\lambda_{ps, c} = \frac{\lambda_{cn, c}}{n} \\
    \end{array}
\end{equation}
From the previous formulas and the open queue model we can compute the Service Demand, Utilization factor and Response time.
\begin{equation}
    \begin{array}{l}
        D_{ps, t} = V_{ps, t} * S_{ps, t} \\
        D_{ps, e} = V_{ps, e} * S_{ps, e} \\
        D_{ps, c} = V_{ps, c} * S_{ps, c} \\
        U_{ps, t} = \lambda_{ps, t} * D_{ps, t} \\
        U_{ps, e} = \lambda_{ps, e} * D_{ps, e} \\
        U_{ps, c} = \lambda_{ps, c} * D_{ps, c} \\
        U_{ps} = U_{ps, t} + U_{ps, e} + U_{ps, c} \\
        R_{ps, t} = \frac{D_{ps, t}}{1 - U_{ps}} \\
        R_{ps, e} = \frac{D_{ps, e}}{1 - U_{ps}} \\
        R_{ps, c} = \frac{D_{ps, c}}{1 - U_{ps}} \\
    \end{array}
\end{equation}

\subsection{Regional Node}
The rates are the same as the Primary-Secondary LAN, since all the elements that go through that LAN will need to go through the relative Regional Node.\\
\begin{equation}
	\begin{array}{l}
		V_{rn, t} = 1 \\
		V_{rn, e} = 2 \\ %reply
		V_{rn,c} = 1 \\
		\lambda_{rn, t} = \lambda_{ps, t} \\
		\lambda_{rn, e} = \lambda_{ps, e} \\
		\lambda_{rn, c} = \frac{\lambda_{cn, c}}{n} \\
	\end{array}
\end{equation}
From the previous formulas and the open queue model we can compute the Service Demand, Utilization factor and Response time.
\begin{equation}
	\begin{array}{l}
		D_{rn, t} = V_{rn, t} * S_{rn, t} \\
		D_{rn, e} = V_{rn, e} * S_{rn, e} \\
		D_{rn, c} = V_{rn, c} * S_{rn, c} \\
		U_{rn, t} = \lambda_{rn, t} * D_{rn, t} \\
		U_{rn, e} = \lambda_{rn, e} * D_{rn, e} \\
		U_{rn, c} = \lambda_{rn, c} * D_{rn, c} \\
		U_{rn} = U_{rn, t} + U_{rn, e} + U_{rn, c} \\
		R_{rn, t} = \frac{D_{rn, t}}{1 - U_{rn}} \\
		R_{rn, e} = \frac{D_{rn, e}}{1 - U_{rn}} \\
		R_{rn, c} = \frac{D_{rn, c}}{1 - U_{rn}} \\
	\end{array}
\end{equation}

\subsection{Regional LAN}
\begin{figure}[H]
	\hspace*{-3.75cm}
	\centering
	\frame{\includesvg[width=20cm, height=17cm]{queue_network_regional_lan}}
	\caption{Focus on regional LAN and regional nodes}
\end{figure}
The regional LAN connects the central node to all the regionals, for this reason the rate of transition and telemetry will be multiplied by $n$, the number of regional node.\\
In particular, regarding the telemetry events the regional node rates will also be divided by the aggregation factor, made by those nodes which will cut the number of messages that will actually be sent up the LAN.\\
The command rate is not divided by anything since all the commands will need to go through this LAN.
\begin{equation}
	\begin{array}{l}
		V_{rl, t} = 1 \\
		V_{rl, e} = 2 \\ %reply
		V_{rl,c} = 1 \\
		\lambda_{rl, t} = n*\frac{\lambda_{rn, t}}{aggr_{rn}} \\
		\lambda_{rl, e} = n*\lambda_{rn, e} \\
		\lambda_{rl, c} = \lambda_{cn, c} \\
	\end{array}
\end{equation}
From the previous formulas and the open queue model we can compute the Service Demand, Utilization factor and Response time.
\begin{equation}
	\begin{array}{l}
		D_{rl, t} = V_{rl, t} * S_{rl, t} \\
		D_{rl, e} = V_{rl, e} * S_{rl, e} \\
		D_{rl, c} = V_{rl, c} * S_{rl, c} \\
		U_{rl, t} = \lambda_{rl, t} * D_{rl, t} \\
		U_{rl, e} = \lambda_{rl, e} * D_{rl, e} \\
		U_{rl, c} = \lambda_{rl, c} * D_{rl, c} \\
		U_{rl} = U_{rl, t} + U_{rl, e} + U_{rl, c} \\
		R_{rl, t} = \frac{D_{rl, t}}{1 - U_{rl}} \\
		R_{rl, e} = \frac{D_{rl, e}}{1 - U_{rl}} \\
		R_{rl, c} = \frac{D_{rl, c}}{1 - U_{rl}} \\
	\end{array}
\end{equation}
Regarding the WAN CENTRAL, it forwards all messages from the CENTRAL LAN to the REGIONAL LAN. The REGIONAL LAN instead, forwards all the messages from the regional nodes to the WAN CENTRAL, so the service time and number of visits per message are the same, with the exception of the service time which can be different.
\subsection{Central LAN}
\begin{figure}[H]
	\hspace*{-3.75cm}
	\centering
	\frame{\includesvg[width=20cm, height=17cm]{queue_network_central_lan}}
	\caption{Focus on regional LAN, WAN central, central LAN and central node}
\end{figure}
The Central LAN has the same traffic as the Regional LAN, with the only different value being the service time.\\
The reasoning behind this LAN's existence is that there may be more than one central node, by the specific there can be two for fault tolerance and availability.
\begin{equation}
	\begin{array}{l}
		V_{cl, t} = 1 \\
		V_{cl, e} = 2 \\ %reply
		V_{cl,c} = 1 \\
		\lambda_{cl, t} = \lambda_{rl,t} \\
		\lambda_{cl, e} = \lambda_{rl,e} \\
		\lambda_{cl, c} = \lambda_{cn, c} \\
	\end{array}
\end{equation}
From the previous formulas and the open queue model we can compute the Service Demand, Utilization factor and Response time.
\begin{equation}
	\begin{array}{l}
		D_{cl, t} = V_{cl, t} * S_{cl, t} \\
		D_{cl, e} = V_{cl, e} * S_{cl, e} \\
		D_{cl, c} = V_{cl, c} * S_{cl, c} \\
		U_{cl, t} = \lambda_{cl, t} * D_{cl, t} \\
		U_{cl, e} = \lambda_{cl, e} * D_{cl, e} \\
		U_{cl, c} = \lambda_{cl, c} * D_{cl, c} \\
		U_{cl} = U_{cl, t} + U_{cl, e} + U_{cl, c} \\
		R_{cl, t} = \frac{D_{cl, t}}{1 - U_{cl}} \\
		R_{cl, e} = \frac{D_{cl, e}}{1 - U_{cl}} \\
		R_{cl, c} = \frac{D_{cl, c}}{1 - U_{cl}} \\
	\end{array}
\end{equation}
\subsection{Central Node}
In the central node $\lambda_{cn,c}$ is given by the user sending commands to adjust the network(or an automatic system) thus it would just be a parameter.
\begin{equation}
	\begin{array}{l}
		V_{cn, t} = 1 \\
		V_{cn, e} = 2 \\ %reply
		V_{cn,c} = 1 \\
		\lambda_{cn, t} = \lambda_{cl,t} \\
		\lambda_{cn, e} = \lambda_{cl,e} \\
	\end{array}
\end{equation}
From the previous formulas and the open queue model we can compute the Service Demand, Utilization factor and Response time.
\begin{equation}
	\begin{array}{l}
		D_{cn, t} = V_{cn, t} * S_{cn, t} \\
		D_{cn, e} = V_{cn, e} * S_{cn, e} \\
		D_{cn, c} = V_{cn, c} * S_{cn, c} \\
		U_{cn, t} = \lambda_{cn, t} * D_{cn, t} \\
		U_{cn, e} = \lambda_{cn, e} * D_{cn, e} \\
		U_{cn, c} = \lambda_{cn, c} * D_{cn, c} \\
		U_{cn} = U_{cn, t} + U_{cn, e} + U_{cn, c} \\
		R_{cn, t} = \frac{D_{cn, t}}{1 - U_{cn}} \\
		R_{cn, e} = \frac{D_{cn, e}}{1 - U_{cn}} \\
		R_{cn, c} = \frac{D_{cn, c}}{1 - U_{cn}} \\
	\end{array}
\end{equation}

\subsubsection{Central node storage system}
In this first draft, the central node storage system is not characterized with a specific system(RAID 1-5, etc) and it is modeled as a simple queue filled by the central node.\\
Since the central node will do some aggregation and computations on the telemetry messages received by the regional nodes, the rate arriving on the disk will be divided by the aggregation parameter of the central node.\\
In this model the commands go directly to the lower levels, in a future extension of the project they may need to be logged onto the disk and thus we will need a $\lambda_{cns, c}$.

\begin{equation}
	\begin{array}{l}
		V_{cns, t} = 1 \\
		V_{cns, e} = 1 \\
		\lambda_{cns, t} = \frac{\lambda_{cn,t}}{aggr_{cn}} \\
		\lambda_{cns, e} = \lambda_{cn,e} \\
	\end{array}
\end{equation}
From the previous formulas and the open queue model we can compute the Service Demand, Utilization factor and Response time.
\begin{equation}
	\begin{array}{l}
		D_{cn, t} = V_{cn, t} * S_{cn, t} \\
		D_{cn, e} = V_{cn, e} * S_{cn, e} \\
		U_{cn, t} = \lambda_{cn, t} * D_{cn, t} \\
		U_{cn, e} = \lambda_{cn, e} * D_{cn, e} \\
		U_{cn} = U_{cn, t} + U_{cn, e} \\
		R_{cn, t} = \frac{D_{cn, t}}{1 - U_{cn}} \\
		R_{cn, e} = \frac{D_{cn, e}}{1 - U_{cn}} \\
	\end{array}
\end{equation}

\end{document}
